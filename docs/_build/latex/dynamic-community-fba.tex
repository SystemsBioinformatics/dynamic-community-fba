%% Generated by Sphinx.
\def\sphinxdocclass{report}
\documentclass[letterpaper,10pt,english]{sphinxmanual}
\ifdefined\pdfpxdimen
   \let\sphinxpxdimen\pdfpxdimen\else\newdimen\sphinxpxdimen
\fi \sphinxpxdimen=.75bp\relax
\ifdefined\pdfimageresolution
    \pdfimageresolution= \numexpr \dimexpr1in\relax/\sphinxpxdimen\relax
\fi
%% let collapsible pdf bookmarks panel have high depth per default
\PassOptionsToPackage{bookmarksdepth=5}{hyperref}

\PassOptionsToPackage{booktabs}{sphinx}
\PassOptionsToPackage{colorrows}{sphinx}

\PassOptionsToPackage{warn}{textcomp}
\usepackage[utf8]{inputenc}
\ifdefined\DeclareUnicodeCharacter
% support both utf8 and utf8x syntaxes
  \ifdefined\DeclareUnicodeCharacterAsOptional
    \def\sphinxDUC#1{\DeclareUnicodeCharacter{"#1}}
  \else
    \let\sphinxDUC\DeclareUnicodeCharacter
  \fi
  \sphinxDUC{00A0}{\nobreakspace}
  \sphinxDUC{2500}{\sphinxunichar{2500}}
  \sphinxDUC{2502}{\sphinxunichar{2502}}
  \sphinxDUC{2514}{\sphinxunichar{2514}}
  \sphinxDUC{251C}{\sphinxunichar{251C}}
  \sphinxDUC{2572}{\textbackslash}
\fi
\usepackage{cmap}
\usepackage[T1]{fontenc}
\usepackage{amsmath,amssymb,amstext}
\usepackage{babel}



\usepackage{tgtermes}
\usepackage{tgheros}
\renewcommand{\ttdefault}{txtt}



\usepackage[Bjarne]{fncychap}
\usepackage{sphinx}

\fvset{fontsize=auto}
\usepackage{geometry}


% Include hyperref last.
\usepackage{hyperref}
% Fix anchor placement for figures with captions.
\usepackage{hypcap}% it must be loaded after hyperref.
% Set up styles of URL: it should be placed after hyperref.
\urlstyle{same}


\usepackage{sphinxmessages}
\setcounter{tocdepth}{1}



\title{dynamic\sphinxhyphen{}community\sphinxhyphen{}fba}
\date{Jul 05, 2023}
\release{}
\author{unknown}
\newcommand{\sphinxlogo}{\vbox{}}
\renewcommand{\releasename}{}
\makeindex
\begin{document}

\ifdefined\shorthandoff
  \ifnum\catcode`\=\string=\active\shorthandoff{=}\fi
  \ifnum\catcode`\"=\active\shorthandoff{"}\fi
\fi

\pagestyle{empty}
\sphinxmaketitle
\pagestyle{plain}
\sphinxtableofcontents
\pagestyle{normal}
\phantomsection\label{\detokenize{index::doc}}


\sphinxAtStartPar
Welcome to the documentation of dynamic\sphinxhyphen{}community\sphinxhyphen{}fba. The documentation provides you with a comprehensive guidance on using the dynamic\sphinxhyphen{}community\sphinxhyphen{}fba Python package for modeling dynamic flux balance analysis (FBA) in microbial consortia. By utilizing the Genome Scale Metabolic Models (GSMM’s) of your organisms of interest, our package enables you to analyze and explore the intricate interactions among two or more organisms.

\sphinxAtStartPar
The package covers three different ways for modelling microbial consortia’s:
\begin{itemize}
\item {} 
\sphinxAtStartPar
Parallel FBA {[}refference to paper{]}

\item {} 
\sphinxAtStartPar
Joint dynamic FBA {[}refference to paper{]}

\item {} 
\sphinxAtStartPar
endPointFBA

\end{itemize}

\sphinxAtStartPar
The documentation is designed to assist users in understanding the concepts, usage, and implementation of these dynamic FBA techniques. One of the key features of dynamic\sphinxhyphen{}community\sphinxhyphen{}fba is the ability to construct and export a community matrix, which represents the joint stoichiometry matrix of the provided GSMM’s models. The documentation provides in\sphinxhyphen{}depth explanations on how to build and utilize this matrix to study the dynamics of multi\sphinxhyphen{}organism interactions and metabolic networks.

\sphinxAtStartPar
To ensure a smooth start, the documentation includes a section outlining the prerequisites and installation guide. Users will find step\sphinxhyphen{}by\sphinxhyphen{}step instructions on installing the necessary dependencies and setting up the dynamic\sphinxhyphen{}community\sphinxhyphen{}fba package in their Python environment. Additionally, the documentation highlights the compatibility requirements and recommends best practices for a successful installation.
By following the documentation, users will gain a comprehensive understanding of the dynamic\sphinxhyphen{}community\sphinxhyphen{}fba package and its capabilities. They will learn how to leverage its functionalities to perform Parallel FBA, joint FBA, and endPointFBA analyses, enabling them to study the dynamic behavior of metabolic networks in diverse biological systems.

\sphinxAtStartPar
The dynamic\sphinxhyphen{}community\sphinxhyphen{}fba documentation serves as a valuable resource for researchers, scientists, and students working in the field of systems biology, metabolic engineering, and microbial ecology. It empowers users to effectively model and analyze dynamic flux balance analyses, facilitating a deeper understanding of the complex interactions between organisms in various biological systems.

\sphinxstepscope


\chapter{1. Installation and Requirements}
\label{\detokenize{1_installation/home:installation-and-requirements}}\label{\detokenize{1_installation/home::doc}}

\section{Prerequisites}
\label{\detokenize{1_installation/home:prerequisites}}
\sphinxAtStartPar
The dynamic\sphinxhyphen{}community\sphinxhyphen{}fba package relies on the \sphinxtitleref{cbmpy} library %
\begin{footnote}[1]\sphinxAtStartFootnote
PySCeS Constraint Based Modelling (\sphinxurl{http://cbmpy.sourceforge.net}) Copyright (C) 2010\sphinxhyphen{}2023 Brett G. Olivier, Vrije Universiteit Amsterdam, Amsterdam, The Netherlands
%
\end{footnote} for handling constraint\sphinxhyphen{}based metabolic models. Like
\sphinxhref{https://opencobra.github.io/cobrapy/\#:~:text=cobrapy\%20is\%20a\%20python\%20package,io.}{cobrapy} \sphinxtitleref{cbmpy\textasciigrave{}} is a Python package that simplifies the creation, loading, and manipulation of constraint\sphinxhyphen{}based models, allowing interaction with LP\sphinxhyphen{}solvers like ‘cplex’ or ‘GLPK’.

\sphinxAtStartPar
To install \sphinxtitleref{cbmpy}, you can use the following command:

\begin{sphinxVerbatim}[commandchars=\\\{\}]
pip\PYG{+w}{ }install\PYG{+w}{ }cbmpy
\end{sphinxVerbatim}

\sphinxAtStartPar
For more information and detailed documentation on using \sphinxtitleref{cbmpy}, please refer to the \sphinxhref{https://github.com/SystemsBioinformatics/cbmpy}{cbmpy GitHub repository} and the \sphinxhref{https://pythonhosted.org/cbmpy/modules\_doc.html}{cbmpy documentation}.


\section{Dynamic Community FBA}
\label{\detokenize{1_installation/home:dynamic-community-fba}}
\sphinxAtStartPar
After the installation of \sphinxtitleref{cbmpy} you can install dynamic community FBA using the following command

\begin{sphinxVerbatim}[commandchars=\\\{\}]
pip\PYG{+w}{ }install\PYG{+w}{ }NAME
\end{sphinxVerbatim}

\sphinxAtStartPar
More text if needed about the installation


\section{Escher}
\label{\detokenize{1_installation/home:escher}}
\sphinxAtStartPar
Maybe we can write some easy converter functions for known maps. To display the models
have to think about this

\sphinxstepscope


\chapter{2. Getting Started}
\label{\detokenize{2_getting_started/home:getting-started}}\label{\detokenize{2_getting_started/home::doc}}
\sphinxAtStartPar
With the package a the models for \sphinxstyleemphasis{E. coli core metabolism}, \sphinxstyleemphasis{Streptococcus thermophilus} and \sphinxstyleemphasis{Lactobacillus delbrueckii}, here
we will show you the basic operations for loading and modifying a cbmpy model. If you’ve worked with \sphinxtitleref{cobrapy} previously
you will see \sphinxtitleref{cbmpy} is not that different. If this is your first time working with GSMM’s and FBA we recommend  you to start with
the extensive cbmpy tutorial (if we have completed this)


\section{2.1. Loading a model using cbmpy}
\label{\detokenize{2_getting_started/home:loading-a-model-using-cbmpy}}
\sphinxAtStartPar
To load a model and perform a simple FBA analysis on it type:

\begin{sphinxVerbatim}[commandchars=\\\{\}]
\PYG{k+kn}{import} \PYG{n+nn}{cbmpy}
\PYG{k+kn}{from} \PYG{n+nn}{cbmpy}\PYG{n+nn}{.}\PYG{n+nn}{CBModel} \PYG{k+kn}{import} \PYG{n}{Model}

\PYG{n}{model}\PYG{p}{:} \PYG{n}{Model} \PYG{o}{=} \PYG{n}{cbmpy}\PYG{o}{.}\PYG{n}{loadModel}\PYG{p}{(}\PYG{l+s+s2}{\PYGZdq{}}\PYG{l+s+s2}{data/bigg\PYGZus{}models/e\PYGZus{}coli\PYGZus{}core.xml}\PYG{l+s+s2}{\PYGZdq{}}\PYG{p}{)}
\PYG{n}{cbmpy}\PYG{o}{.}\PYG{n}{doFBA}\PYG{p}{(}\PYG{n}{model}\PYG{p}{)}
\PYG{n}{FBAsol} \PYG{o}{=} \PYG{n}{model}\PYG{o}{.}\PYG{n}{getSolutionVector}\PYG{p}{(}\PYG{n}{names}\PYG{o}{=}\PYG{k+kc}{True}\PYG{p}{)}
\PYG{n}{FBAsol} \PYG{o}{=} \PYG{n+nb}{dict}\PYG{p}{(}\PYG{n+nb}{zip}\PYG{p}{(}\PYG{n}{FBAsol}\PYG{p}{[}\PYG{l+m+mi}{1}\PYG{p}{]}\PYG{p}{,} \PYG{n}{FBAsol}\PYG{p}{[}\PYG{l+m+mi}{0}\PYG{p}{]}\PYG{p}{)}\PYG{p}{)}
\end{sphinxVerbatim}

\sphinxAtStartPar
Here the model refers to a \sphinxcode{\sphinxupquote{cbmpy.CBModel}} which represents the GSMM of the loaded organism .


\section{2.2. SBML and cobra models}
\label{\detokenize{2_getting_started/home:sbml-and-cobra-models}}
\sphinxAtStartPar
The \sphinxcode{\sphinxupquote{cbmpy.load\_model()}} function is designed to efficiently handle a wide range of models. It seamlessly supports the
import of models encoded in the standardized \sphinxhref{https://sbml.org/}{Systems Biology Markup Language (SBML)} format, as well as models exported by
\sphinxtitleref{cobrapy}. This means that you can easily work with different versions of SBML and \sphinxtitleref{cobrapy} models without having to
specify them explicitly. This flexibility simplifies the model loading process.

\begin{sphinxadmonition}{note}{Note:}
\sphinxAtStartPar
Sometimes the conversion of exchanges, sinks or other boundary conditions are not properly set when exporting or importing
a \sphinxtitleref{cobra} model into \sphinxtitleref{cbmpy} therefore always check if these reactions are set correctly in the loaded model.*
\end{sphinxadmonition}


\section{2.3. Saving a model}
\label{\detokenize{2_getting_started/home:saving-a-model}}
\sphinxAtStartPar
There are two ways to save a cbmpy model. The easiest way is to save your altered model to the latest version of SBML:

\begin{sphinxVerbatim}[commandchars=\\\{\}]
\PYG{n}{reaction} \PYG{o}{=} \PYG{n}{model}\PYG{o}{.}\PYG{n}{getReaction}\PYG{p}{(}\PYG{l+s+s2}{\PYGZdq{}}\PYG{l+s+s2}{R\PYGZus{}EX\PYGZus{}glc\PYGZus{}\PYGZus{}D\PYGZus{}e}\PYG{l+s+s2}{\PYGZdq{}}\PYG{p}{)} \PYG{c+c1}{\PYGZsh{}Get a reaction from the model}
\PYG{n}{reaction}\PYG{o}{.}\PYG{n}{setLowerBound}\PYG{p}{(}\PYG{l+m+mi}{0}\PYG{p}{)} \PYG{c+c1}{\PYGZsh{}Alter the reaction in the model}

\PYG{n}{cbmpy}\PYG{o}{.}\PYG{n}{saveModel}\PYG{p}{(}\PYG{n}{model}\PYG{p}{,} \PYG{l+s+s2}{\PYGZdq{}}\PYG{l+s+s2}{adjusted\PYGZus{}model.xml}\PYG{l+s+s2}{\PYGZdq{}}\PYG{p}{)} \PYG{c+c1}{\PYGZsh{}Save the new model to a XML file}
\end{sphinxVerbatim}

\sphinxAtStartPar
If you don’t want to save the model to to the latest version of SBML or you wan’t to save it to a \sphinxtitleref{cobrapy} model you can call one of the below functions:

\begin{sphinxVerbatim}[commandchars=\\\{\}]
\PYG{n}{cbmpy}\PYG{o}{.}\PYG{n}{writeCOBRASBML}\PYG{p}{(}\PYG{o}{.}\PYG{o}{.}\PYG{o}{.}\PYG{p}{)}
\PYG{n}{cbmpy}\PYG{o}{.}\PYG{n}{writeFVAtoCSV}\PYG{p}{(}\PYG{o}{.}\PYG{o}{.}\PYG{o}{.}\PYG{p}{)}
\PYG{n}{cbmpy}\PYG{o}{.}\PYG{n}{writeModelToExcel97}\PYG{p}{(}\PYG{o}{.}\PYG{o}{.}\PYG{o}{.}\PYG{p}{)}
\PYG{n}{cbmpy}\PYG{o}{.}\PYG{n}{writeSBML3FBCV2}\PYG{p}{(}\PYG{o}{.}\PYG{o}{.}\PYG{o}{.}\PYG{p}{)}
\end{sphinxVerbatim}


\section{2.4. Reactions, Reagents and Species}
\label{\detokenize{2_getting_started/home:reactions-reagents-and-species}}
\sphinxAtStartPar
In \sphinxtitleref{cbmpy}, the \sphinxcode{\sphinxupquote{cbmpy.CBModel}} object forms the basis of the model. When working with the model,
most modifications will involve manipulating this object. In the previous section,
we demonstrated how to load the \sphinxstyleemphasis{E. coli core metabolism}. Now, let’s explore some basic alterations that can be made
to the model. For a more comprehensive understanding of the functionalities available in the \sphinxcode{\sphinxupquote{cbmpy.CBModel}} object and
other features of cbmpy, we recommend referring to the extensive \sphinxhref{https://pythonhosted.org/cbmpy/modules\_doc.html}{documentation}.


\subsection{Reactions}
\label{\detokenize{2_getting_started/home:reactions}}
\sphinxAtStartPar
To list all the reactions in the model, or list the reaction containing a certain string you can call the following functions:

\begin{sphinxVerbatim}[commandchars=\\\{\}]
\PYG{n}{modelRxns} \PYG{o}{=} \PYG{n}{model}\PYG{o}{.}\PYG{n}{getReactionIds}\PYG{p}{(}\PYG{p}{)} \PYG{c+c1}{\PYGZsh{}All the reactions, as a list[str]}
\PYG{n+nb}{print}\PYG{p}{(}\PYG{n}{modelRxns}\PYG{p}{)}

\PYG{n}{model}\PYG{o}{.}\PYG{n}{getReactionIds}\PYG{p}{(}\PYG{l+s+s1}{\PYGZsq{}}\PYG{l+s+s1}{PG}\PYG{l+s+s1}{\PYGZsq{}}\PYG{p}{)} \PYG{c+c1}{\PYGZsh{}Outputs only reactions with \PYGZdq{}PG\PYGZdq{} in their ID}
\end{sphinxVerbatim}

\sphinxAtStartPar
Once you have identified your reaction of interest, you can easily access its key details, including the reagents, upper and lower bounds, and equation, as follows:

\begin{sphinxVerbatim}[commandchars=\\\{\}]
\PYG{k+kn}{from} \PYG{n+nn}{cbmpy}\PYG{n+nn}{.}\PYG{n+nn}{CBModel} \PYG{k+kn}{import} \PYG{n}{Reaction}\PYG{p}{,} \PYG{n}{Reagent}\PYG{p}{,} \PYG{n}{Species}

\PYG{n}{reaction}\PYG{p}{:} \PYG{n}{Reaction} \PYG{o}{=} \PYG{n}{model}\PYG{o}{.}\PYG{n}{getReaction}\PYG{p}{(}\PYG{l+s+s2}{\PYGZdq{}}\PYG{l+s+s2}{R\PYGZus{}PGK}\PYG{l+s+s2}{\PYGZdq{}}\PYG{p}{)}

\PYG{n}{reagents}\PYG{p}{:} \PYG{n+nb}{list}\PYG{p}{[}\PYG{n}{Reagent}\PYG{p}{]} \PYG{o}{=} \PYG{n}{reaction}\PYG{o}{.}\PYG{n}{getReagentObjIds}\PYG{p}{(}\PYG{p}{)}  \PYG{c+c1}{\PYGZsh{} Get all reagent ids of the reaction}
\PYG{n+nb}{print}\PYG{p}{(}\PYG{n}{reagents}\PYG{p}{)}

\PYG{n}{bounds} \PYG{o}{=} \PYG{p}{[}\PYG{n}{reaction}\PYG{o}{.}\PYG{n}{getLowerBound}\PYG{p}{(}\PYG{p}{)}\PYG{p}{,} \PYG{n}{reaction}\PYG{o}{.}\PYG{n}{getUpperBound}\PYG{p}{(}\PYG{p}{)}\PYG{p}{]} \PYG{c+c1}{\PYGZsh{} Get the lower and upper bound}
\PYG{n+nb}{print}\PYG{p}{(}\PYG{n}{bounds}\PYG{p}{)}

\PYG{n}{equation} \PYG{o}{=} \PYG{n}{reaction}\PYG{o}{.}\PYG{n}{getEquation}\PYG{p}{(}\PYG{p}{)} \PYG{c+c1}{\PYGZsh{} Get the reactions equation}
\PYG{n+nb}{print}\PYG{p}{(}\PYG{n}{equation}\PYG{p}{)}
\end{sphinxVerbatim}

\sphinxAtStartPar
Furthermore you can check if a reaction is reversible and if it is an exchange reaction:

\begin{sphinxVerbatim}[commandchars=\\\{\}]
\PYG{n+nb}{print}\PYG{p}{(}\PYG{n}{reaction}\PYG{o}{.}\PYG{n}{is\PYGZus{}exchange}\PYG{p}{)} \PYG{c+c1}{\PYGZsh{}True if the reaction is an exchange reaction}

\PYG{n+nb}{print}\PYG{p}{(}\PYG{n}{reaction}\PYG{o}{.}\PYG{n}{reversible}\PYG{p}{)} \PYG{c+c1}{\PYGZsh{}True if the reaction is reversible}
\end{sphinxVerbatim}

\sphinxAtStartPar
You can easily add your own defined reactions to the model using \sphinxcode{\sphinxupquote{model.createReaction()}}, if we for example want to add the
reaction: \sphinxcode{\sphinxupquote{ATP + H2O \sphinxhyphen{}\textgreater{} ADP + Pi}} we can do this with the following code:

\begin{sphinxVerbatim}[commandchars=\\\{\}]
\PYG{n}{model}\PYG{o}{.}\PYG{n}{createReaction}\PYG{p}{(}\PYG{l+s+s1}{\PYGZsq{}}\PYG{l+s+s1}{ATPsink}\PYG{l+s+s1}{\PYGZsq{}}\PYG{p}{,} \PYG{n}{reversible} \PYG{o}{=} \PYG{k+kc}{False}\PYG{p}{)} \PYG{c+c1}{\PYGZsh{} Create a new empty irreversible reaction}

\PYG{c+c1}{\PYGZsh{} Add the reagents to the reaction, All metabolites already existed in the model so we did not}
\PYG{c+c1}{\PYGZsh{} Need to create them}
\PYG{n}{model}\PYG{o}{.}\PYG{n}{createReactionReagent}\PYG{p}{(}\PYG{l+s+s1}{\PYGZsq{}}\PYG{l+s+s1}{ATPsink}\PYG{l+s+s1}{\PYGZsq{}}\PYG{p}{,} \PYG{n}{metabolite} \PYG{o}{=} \PYG{l+s+s2}{\PYGZdq{}}\PYG{l+s+s2}{M\PYGZus{}atp\PYGZus{}c}\PYG{l+s+s2}{\PYGZdq{}} \PYG{p}{,} \PYG{n}{coefficient} \PYG{o}{=} \PYG{o}{\PYGZhy{}}\PYG{l+m+mi}{1}\PYG{p}{)}
\PYG{n}{model}\PYG{o}{.}\PYG{n}{createReactionReagent}\PYG{p}{(}\PYG{l+s+s1}{\PYGZsq{}}\PYG{l+s+s1}{ATPsink}\PYG{l+s+s1}{\PYGZsq{}}\PYG{p}{,} \PYG{n}{metabolite} \PYG{o}{=} \PYG{l+s+s2}{\PYGZdq{}}\PYG{l+s+s2}{M\PYGZus{}adp\PYGZus{}c}\PYG{l+s+s2}{\PYGZdq{}}\PYG{p}{,} \PYG{n}{coefficient} \PYG{o}{=}\PYG{l+m+mi}{1}\PYG{p}{)}
\PYG{n}{model}\PYG{o}{.}\PYG{n}{createReactionReagent}\PYG{p}{(}\PYG{l+s+s1}{\PYGZsq{}}\PYG{l+s+s1}{ATPsink}\PYG{l+s+s1}{\PYGZsq{}}\PYG{p}{,} \PYG{n}{metabolite} \PYG{o}{=}  \PYG{l+s+s2}{\PYGZdq{}}\PYG{l+s+s2}{M\PYGZus{}h2o\PYGZus{}c}\PYG{l+s+s2}{\PYGZdq{}}\PYG{p}{,} \PYG{n}{coefficient} \PYG{o}{=} \PYG{o}{\PYGZhy{}}\PYG{l+m+mi}{1}\PYG{p}{)}
\PYG{n}{model}\PYG{o}{.}\PYG{n}{createReactionReagent}\PYG{p}{(}\PYG{l+s+s1}{\PYGZsq{}}\PYG{l+s+s1}{ATPsink}\PYG{l+s+s1}{\PYGZsq{}}\PYG{p}{,} \PYG{n}{metabolite} \PYG{o}{=} \PYG{l+s+s2}{\PYGZdq{}}\PYG{l+s+s2}{M\PYGZus{}pi\PYGZus{}c}\PYG{l+s+s2}{\PYGZdq{}} \PYG{p}{,} \PYG{n}{coefficient} \PYG{o}{=} \PYG{l+m+mi}{1}\PYG{p}{)}
\end{sphinxVerbatim}


\subsection{Reagents}
\label{\detokenize{2_getting_started/home:reagents}}
\sphinxAtStartPar
Tohe \sphinxcode{\sphinxupquote{Reagent}} class  represents a reagent within a reaction, providing essential information about its properties and characteristics.
Within the class, users can access and manipulate the reagents associated with a specific reaction within the model. The reagent itself
is linked to a \sphinxcode{\sphinxupquote{Species}} which we will cover shortly.
You can access a reagent by retrieving it from an instance of the \sphinxcode{\sphinxupquote{Reaction}} class, given the \sphinxtitleref{R\_PGK} reaction from the previous example
we can access information about a reagent as follows:

\begin{sphinxVerbatim}[commandchars=\\\{\}]
\PYG{n}{reagent}\PYG{p}{:} \PYG{n}{Reagent} \PYG{o}{=} \PYG{n}{reaction}\PYG{o}{.}\PYG{n}{getReagent}\PYG{p}{(}\PYG{l+s+s2}{\PYGZdq{}}\PYG{l+s+s2}{R\PYGZus{}PGK\PYGZus{}M\PYGZus{}3pg\PYGZus{}c}\PYG{l+s+s2}{\PYGZdq{}}\PYG{p}{)}

\PYG{n}{reagent}\PYG{o}{.}\PYG{n}{getCoefficient}\PYG{p}{(}\PYG{p}{)} \PYG{c+c1}{\PYGZsh{} Get the reagent\PYGZsq{}s stoichiometric coefficient}

\PYG{n}{reagent}\PYG{o}{.}\PYG{n}{getCompartmentId}\PYG{p}{(}\PYG{p}{)} \PYG{c+c1}{\PYGZsh{}Get the compartment}

\PYG{n}{reagent}\PYG{o}{.}\PYG{n}{getSpecies}\PYG{p}{(}\PYG{p}{)} \PYG{c+c1}{\PYGZsh{} Get the species id corresponding to this reagent}
\end{sphinxVerbatim}

\sphinxAtStartPar
If a reagent has a negative coefficient it is consumed by the reaction, if the reagent has a positive coefficient it is created by the reaction.


\subsection{Species}
\label{\detokenize{2_getting_started/home:species}}
\sphinxAtStartPar
Species represent the metabolites in the system using the \sphinxcode{\sphinxupquote{Species}} object you can easily retrieve details such as the species’ molecular formula, charge, and compartment information.
Furthermore you can list the reactions in which a species is consumed or created

\begin{sphinxVerbatim}[commandchars=\\\{\}]
\PYG{n}{species}\PYG{p}{:} \PYG{n}{Species} \PYG{o}{=} \PYG{n}{model}\PYG{o}{.}\PYG{n}{getSpecies}\PYG{p}{(}\PYG{l+s+s2}{\PYGZdq{}}\PYG{l+s+s2}{M\PYGZus{}pi\PYGZus{}c}\PYG{l+s+s2}{\PYGZdq{}}\PYG{p}{)}

\PYG{n}{species}\PYG{o}{.}\PYG{n}{getChemFormula}\PYG{p}{(}\PYG{p}{)}
\PYG{n}{species}\PYG{o}{.}\PYG{n}{getCharge}\PYG{p}{(}\PYG{p}{)}
\PYG{n}{species}\PYG{o}{.}\PYG{n}{getCompartmentId}\PYG{p}{(}\PYG{p}{)} \PYG{c+c1}{\PYGZsh{} Gives the id of the compartment in which the species lives}
\PYG{n}{species}\PYG{o}{.}\PYG{n}{isReagentOf}\PYG{p}{(}\PYG{p}{)} \PYG{c+c1}{\PYGZsh{} Returns a list of reaction ids in which the species is present}
\end{sphinxVerbatim}


\subsection{Objective function}
\label{\detokenize{2_getting_started/home:objective-function}}
\sphinxAtStartPar
To perform FBA on the model you need to set an objective function. This is the reaction for which the maximal or minimal flux will
be calculated.
To check what the active objective function of the model is you can write:

\begin{sphinxVerbatim}[commandchars=\\\{\}]
\PYG{n}{objective\PYGZus{}ids} \PYG{o}{=} \PYG{n}{model}\PYG{o}{.}\PYG{n}{getActiveObjectiveReactionIds}\PYG{p}{(}\PYG{p}{)}
\PYG{c+c1}{\PYGZsh{}[\PYGZsq{}R\PYGZus{}BIOMASS\PYGZus{}Ecoli\PYGZus{}core\PYGZus{}w\PYGZus{}GAM\PYGZsq{}]}

\PYG{n}{objective} \PYG{o}{=} \PYG{n}{model}\PYG{o}{.}\PYG{n}{getActiveObjective}\PYG{p}{(}\PYG{p}{)}
\PYG{n}{objective}\PYG{o}{.}\PYG{n}{getOperation}\PYG{p}{(}\PYG{p}{)}
\PYG{c+c1}{\PYGZsh{}Maximize}
\end{sphinxVerbatim}

\sphinxAtStartPar
If you would call the function \sphinxcode{\sphinxupquote{cbmpy.doFBA(model)}} FBA will calculate the fluxes such that the flux through the
reaction with id \sphinxtitleref{R\_BIOMASS\_Ecoli\_core\_w\_GAM} is maximal.


\section{2.5. Transitioning to cbmpy from cobrapy}
\label{\detokenize{2_getting_started/home:transitioning-to-cbmpy-from-cobrapy}}
\sphinxAtStartPar
As previously mentioned, any model build using \sphinxtitleref{cobrapy} or any other toolbox for that matter can easily be opened in \sphinxtitleref{cbmpy}. By
just exporting your model of interest to either SBML format or to a cobra model you can import it as an cbmpy model.

\sphinxAtStartPar
Next we will explore how cbmpy models can be used to model the behaviors of microbial communities.

\sphinxstepscope


\chapter{3. Parallel dynamic FBA}
\label{\detokenize{3_parallel_dfba/home:parallel-dynamic-fba}}\label{\detokenize{3_parallel_dfba/home::doc}}
\sphinxstepscope


\chapter{4. The community model}
\label{\detokenize{4_community_matrix/home:the-community-model}}\label{\detokenize{4_community_matrix/home::doc}}

\section{4.1. Definition}
\label{\detokenize{4_community_matrix/home:definition}}
\sphinxAtStartPar
Here we will introduce the concept of the community model, which plays a vital role in upcoming modeling techniques.
In simple terms, the community model represents the combined stoichiometry matrices of N Genome\sphinxhyphen{}Scale Metabolic Models (GSMMs)
for performing Flux Balance Analysis (FBA).

\sphinxAtStartPar
To streamline the handling of multiple models, we have developed the \sphinxcode{\sphinxupquote{CommunityModel}} class. The class offers a structured representation of combined metabolic networks, integrating stoichiometric
information from individual GSMMs. This facilitates in\sphinxhyphen{}depth analysis of complex microbial communities, their dynamics,
and metabolic potentials.

\sphinxAtStartPar
The documentation delves into the underlying principles and rules governing the community model’s construction, ensuring
effective utilization and preventing erroneous model creation. Additionally, we provide practical usage examples to further
illustrate the versatility of the \sphinxcode{\sphinxupquote{CommunityModel}} class.


\section{4.2. Creating and modifying the community model}
\label{\detokenize{4_community_matrix/home:creating-and-modifying-the-community-model}}
\sphinxAtStartPar
To initialize a \sphinxcode{\sphinxupquote{CommunityModel}} you have to give a list of N GSMMs as well as there biomass reaction ids:

\begin{sphinxVerbatim}[commandchars=\\\{\}]
\PYG{k+kn}{import} \PYG{n+nn}{cbmpy}
\PYG{k+kn}{from} \PYG{n+nn}{endPointFBA} \PYG{k+kn}{import} \PYG{n}{CommunityModel}

\PYG{n}{model1} \PYG{o}{=} \PYG{n}{cbmpy}\PYG{o}{.}\PYG{n}{loadModel}\PYG{p}{(}\PYG{l+s+s2}{\PYGZdq{}}\PYG{l+s+s2}{data/bigg\PYGZus{}models/e\PYGZus{}coli\PYGZus{}core.xml}\PYG{l+s+s2}{\PYGZdq{}}\PYG{p}{)}
\PYG{n}{model2} \PYG{o}{=} \PYG{n}{cbmpy}\PYG{o}{.}\PYG{n}{loadModel}\PYG{p}{(}\PYG{l+s+s2}{\PYGZdq{}}\PYG{l+s+s2}{data/bigg\PYGZus{}models/strep\PYGZus{}therm.xml}\PYG{l+s+s2}{\PYGZdq{}}\PYG{p}{)}

\PYG{n}{biomass\PYGZus{}reaction\PYGZus{}model\PYGZus{}1} \PYG{o}{=} \PYG{l+s+s2}{\PYGZdq{}}\PYG{l+s+s2}{R\PYGZus{}BIOMASS\PYGZus{}Ecoli\PYGZus{}core\PYGZus{}w\PYGZus{}GAM}\PYG{l+s+s2}{\PYGZdq{}}
\PYG{n}{biomass\PYGZus{}reaction\PYGZus{}model\PYGZus{}2} \PYG{o}{=} \PYG{l+s+s2}{\PYGZdq{}}\PYG{l+s+s2}{R\PYGZus{}biomass\PYGZus{}STR}\PYG{l+s+s2}{\PYGZdq{}}
\PYG{n}{community\PYGZus{}model}\PYG{p}{:} \PYG{n}{CommunityModel} \PYG{o}{=} \PYG{n}{CommunityModel}\PYG{p}{(}
    \PYG{p}{[}\PYG{n}{model1}\PYG{p}{,} \PYG{n}{model2}\PYG{p}{]}\PYG{p}{,}
    \PYG{p}{[}
        \PYG{n}{biomass\PYGZus{}reaction\PYGZus{}model\PYGZus{}1}\PYG{p}{,}
        \PYG{n}{biomass\PYGZus{}reaction\PYGZus{}model\PYGZus{}2}\PYG{p}{,}
    \PYG{p}{]}\PYG{p}{,}
\PYG{p}{)}
\end{sphinxVerbatim}

\sphinxAtStartPar
If you are not familiar with the biomass reaction ID, there are a couple of ways to identify it.
First, you can look it up in the SBML model. Alternatively, you can check if the active objective function of the model is
set to the biomass reaction by using the following code: \sphinxcode{\sphinxupquote{model.getActiveObjectiveReactionIds()}}.
This will display a list of objective IDs, and you can verify if the biomass reaction is included.

\sphinxAtStartPar
In this example, we have utilized two models included in the package: \sphinxstyleemphasis{E. coli core metabolism}
and \sphinxstyleemphasis{Streptococcus thermophilus}. If desired, you can provide an optional list of alternative IDs to refer to the single models.
This can be done by supplying a list of strings as the third argument to the function.
If no alternative IDs are provided, the community matrix will default to using the model identifiers.
Lastly, you have the option to assign an ID of your own choosing to the new combined model. If not provided the default ID will be used.

\sphinxAtStartPar
It is important to note that the \sphinxcode{\sphinxupquote{CommunityModel}} class is derived from the base \sphinxcode{\sphinxupquote{cbmpy.CBModel}} class, meaning that the
\sphinxcode{\sphinxupquote{CommunityModel}} is still an instance of \sphinxcode{\sphinxupquote{cbmpy.CBModel}}. With the newly initialized object we can obtain even more information.
Check the API for all functionality.

\begin{sphinxadmonition}{warning}{Warning:}
\sphinxAtStartPar
If you want to create a community model of N identical models, it is mandatory to specify the alternative ids.
Without providing alternative IDs, both models would have the same ID during the community model creation process.
As a result, the code would be unable to distinguish between the two models, leading to undesired behavior.
\end{sphinxadmonition}


\section{4.3. Rules}
\label{\detokenize{4_community_matrix/home:rules}}
\sphinxAtStartPar
Maybe move this to an advanced section?

\sphinxAtStartPar
During the initialization of the community model a new \sphinxcode{\sphinxupquote{cbmpy.CBModel}} object is created. In the new object all compartments,
reactions, reagents and species are copied from the provided model. The following considerations were made for each:


\subsection{Compartments:}
\label{\detokenize{4_community_matrix/home:compartments}}
\sphinxAtStartPar
The new model comprises a total of \(\Sigma_{i=1}^{n} (c_i-1) + 1\) compartments, where c represents the number
of compartments in each model i.

\sphinxAtStartPar
In the new model, all compartments from the individual models are copied, except for the \sphinxtitleref{external} or \sphinxtitleref{e} compartment.
The external compartment is reserved to be added at the end. All other compartment ids get a prefix of the id of the model
they belonged to such that we know which compartment corresponds to which organism.

\sphinxAtStartPar
This design ensures that there is only one external compartment in which all species and reactions from all models coexist.

\sphinxAtStartPar
By following this approach, the new model achieves compartmental organization while consolidating all species and reactions
within a unified external compartment.


\subsection{Reactions:}
\label{\detokenize{4_community_matrix/home:reactions}}
\sphinxAtStartPar
Just like the compartments, all reactions are duplicated, and each reaction ID is augmented with its corresponding model ID. This
holds up for all reactions except for the exchange reactions. If two models share an exchange reaction only one is saved in the new
combined model such that we have more control over all.


\subsection{Reagents and species}
\label{\detokenize{4_community_matrix/home:reagents-and-species}}
\sphinxAtStartPar
Before the reactions can be copied to the new model there is a check for which species occur in more than one model.
For these species a new species in the original model is created and all reactions associated to this species have there reagents changed.
By doing so the initial models already have everything set correctly into place to have the reactions copied.

\sphinxAtStartPar
In contrast with the compartment IDs, and the reaction IDs the species IDs are not changed by default. But only if the species id
occurs in two or more models. This is done since we can already quickly lookup to which original model the species belonged by checking
the compartment which it is in.

\begin{sphinxadmonition}{note}{Note:}
\sphinxAtStartPar
It is crucial to verify that the identical reactions and species within different models have consistent IDs before
creating the community model. This is particularly significant for exchange reactions and species localized in the
extracellular space. If these IDs are not uniform, despite referring to the same reactions or species, the CommunityModel
class cannot determine their equivalence accurately.

\sphinxAtStartPar
Please ensure that the corresponding IDs for these reactions and species are harmonized to guarantee the proper
functioning of the CommunityModel.
\end{sphinxadmonition}

\sphinxstepscope


\chapter{5. Dynamic Joint FBA}
\label{\detokenize{5_djoint/home:dynamic-joint-fba}}\label{\detokenize{5_djoint/home::doc}}
\sphinxAtStartPar
The multiple metabolic models of different or the same organism that were combined in the \sphinxcode{\sphinxupquote{CommunityModel}} as described
in the previous chapter can now be used in dynamic joint FBA. The community model incorporates the metabolic reactions and
interactions between the organisms, allowing for the study of their collective behavior and the emergent properties of the
community as a whole.

\sphinxAtStartPar
The joint FBA approach enables the investigation of metabolic exchanges, such as the exchange of nutrients or byproducts,
between organisms within the community. By simulating the community\sphinxhyphen{}level metabolic interactions, researchers can gain
insights into the dependencies, cooperation, competition, and overall dynamics of the organisms in the community.


\section{Joint FBA}
\label{\detokenize{5_djoint/home:joint-fba}}
\sphinxAtStartPar
After defining the \sphinxcode{\sphinxupquote{CommunityModel}} it is easy to refine it in such a way that we can perform a Joint FBA
The only thing left to do is append the biomass reaction of each individual model to create the so called \sphinxtitleref{Community biomass}.
So now we can define the reaction \sphinxcode{\sphinxupquote{X\_c \sphinxhyphen{}\textgreater{} \textasciigrave{} where \textasciigrave{}X\_c}} is the community biomass.

\sphinxAtStartPar
To perform joint FBA run the following:

\begin{sphinxVerbatim}[commandchars=\\\{\}]
\PYG{k+kn}{import} \PYG{n+nn}{cbmpy}
\PYG{k+kn}{from} \PYG{n+nn}{cbmpy}\PYG{n+nn}{.}\PYG{n+nn}{CBModel} \PYG{k+kn}{import} \PYG{n}{Model}
\PYG{k+kn}{from} \PYG{n+nn}{endPointFBA}\PYG{n+nn}{.}\PYG{n+nn}{CommunityModel} \PYG{k+kn}{import} \PYG{n}{CommunityModel}
\PYG{k+kn}{from} \PYG{n+nn}{endPointFBA}\PYG{n+nn}{.}\PYG{n+nn}{DynamicJointFBA} \PYG{k+kn}{import} \PYG{n}{DynamicJointFBA}

\PYG{n}{model1}\PYG{p}{:} \PYG{n}{Model} \PYG{o}{=} \PYG{n}{cbmpy}\PYG{o}{.}\PYG{n}{loadModel}\PYG{p}{(}\PYG{l+s+s2}{\PYGZdq{}}\PYG{l+s+s2}{data/bigg\PYGZus{}models/e\PYGZus{}coli\PYGZus{}core.xml}\PYG{l+s+s2}{\PYGZdq{}}\PYG{p}{)}
\PYG{n}{model\PYGZus{}2} \PYG{o}{=} \PYG{n}{model1}\PYG{o}{.}\PYG{n}{clone}\PYG{p}{(}\PYG{p}{)}

\PYG{n}{combined\PYGZus{}model} \PYG{o}{=} \PYG{n}{CommunityModel}\PYG{p}{(}
    \PYG{p}{[}\PYG{n}{model1}\PYG{p}{,} \PYG{n}{model\PYGZus{}2}\PYG{p}{]}\PYG{p}{,}
    \PYG{p}{[}\PYG{l+s+s2}{\PYGZdq{}}\PYG{l+s+s2}{R\PYGZus{}BIOMASS\PYGZus{}Ecoli\PYGZus{}core\PYGZus{}w\PYGZus{}GAM}\PYG{l+s+s2}{\PYGZdq{}}\PYG{p}{,} \PYG{l+s+s2}{\PYGZdq{}}\PYG{l+s+s2}{R\PYGZus{}BIOMASS\PYGZus{}Ecoli\PYGZus{}core\PYGZus{}w\PYGZus{}GAM}\PYG{l+s+s2}{\PYGZdq{}}\PYG{p}{]}\PYG{p}{,}
    \PYG{p}{[}\PYG{l+s+s2}{\PYGZdq{}}\PYG{l+s+s2}{ecoli\PYGZus{}1}\PYG{l+s+s2}{\PYGZdq{}}\PYG{p}{,} \PYG{l+s+s2}{\PYGZdq{}}\PYG{l+s+s2}{ecoli\PYGZus{}2}\PYG{l+s+s2}{\PYGZdq{}}\PYG{p}{]}\PYG{p}{,}
\PYG{p}{)}  \PYG{c+c1}{\PYGZsh{} Create a CommunityModel of two  E. coli strains competing for resources}


\PYG{c+c1}{\PYGZsh{} Create the joint FBA object with initial biomasses and the initial concentration of glucose}
\PYG{n}{dynamic\PYGZus{}fba} \PYG{o}{=} \PYG{n}{DynamicJointFBA}\PYG{p}{(}
    \PYG{n}{combined\PYGZus{}model}\PYG{p}{,}
    \PYG{p}{[}\PYG{l+m+mf}{0.1}\PYG{p}{,} \PYG{l+m+mf}{0.1}\PYG{p}{]}\PYG{p}{,}
    \PYG{p}{\PYGZob{}}\PYG{l+s+s2}{\PYGZdq{}}\PYG{l+s+s2}{M\PYGZus{}glc\PYGZus{}\PYGZus{}D\PYGZus{}e}\PYG{l+s+s2}{\PYGZdq{}}\PYG{p}{:} \PYG{l+m+mi}{10}\PYG{p}{\PYGZcb{}}\PYG{p}{,}
\PYG{p}{)}

\PYG{c+c1}{\PYGZsh{} Perform FBA on the new joint FBA model object}
\PYG{n}{solution} \PYG{o}{=} \PYG{n}{cbmpy}\PYG{o}{.}\PYG{n}{doFBA}\PYG{p}{(}\PYG{n}{dynamic\PYGZus{}fba}\PYG{o}{.}\PYG{n}{get\PYGZus{}joint\PYGZus{}model}\PYG{p}{(}\PYG{p}{)}\PYG{p}{)}
\PYG{n+nb}{print}\PYG{p}{(}\PYG{n}{solution}\PYG{p}{)}
\end{sphinxVerbatim}


\section{Making it dynamic}
\label{\detokenize{5_djoint/home:making-it-dynamic}}
\sphinxAtStartPar
Dynamic FBA is an extension of the traditional FBA approach that incorporates the element of
time. In dynamic FBA, a specific time step \sphinxtitleref{dt} is selected, and the concentrations of external
metabolites and biomass concentrations are calculated at each time point.
This enables the modeling and analysis of dynamic processes such as metabolic fluxes,
nutrient uptake, and product secretion over time.

\sphinxAtStartPar
The same technique can be applied for using the previously described Joint FBA which we call
\sphinxtitleref{Dynamic Joint FBA}

\sphinxAtStartPar
To perform the joint FBA over time using the \sphinxcode{\sphinxupquote{DynamicJointFBA}} model:

\begin{sphinxVerbatim}[commandchars=\\\{\}]
\PYG{n}{dynamic\PYGZus{}fba}\PYG{o}{.}\PYG{n}{simulate}\PYG{p}{(}\PYG{l+m+mf}{0.1}\PYG{p}{)}
\end{sphinxVerbatim}

\sphinxAtStartPar
You can now easily plot the solution:

\begin{sphinxVerbatim}[commandchars=\\\{\}]
\PYG{k+kn}{import} \PYG{n+nn}{matplotlib}\PYG{n+nn}{.}\PYG{n+nn}{pyplot} \PYG{k}{as} \PYG{n+nn}{plt}

\PYG{n}{ts}\PYG{p}{,} \PYG{n}{metabolites}\PYG{p}{,} \PYG{n}{biomasses} \PYG{o}{=} \PYG{n}{dynamic\PYGZus{}fba}\PYG{o}{.}\PYG{n}{simulate}\PYG{p}{(}\PYG{l+m+mf}{0.1}\PYG{p}{)}


\PYG{n}{ax} \PYG{o}{=} \PYG{n}{plt}\PYG{o}{.}\PYG{n}{subplot}\PYG{p}{(}\PYG{l+m+mi}{1616}\PYG{p}{)}
\PYG{n}{ax}\PYG{o}{.}\PYG{n}{plot}\PYG{p}{(}\PYG{n}{ts}\PYG{p}{,} \PYG{n}{biomasses}\PYG{p}{[}\PYG{l+s+s2}{\PYGZdq{}}\PYG{l+s+s2}{ecoli\PYGZus{}1}\PYG{l+s+s2}{\PYGZdq{}}\PYG{p}{]}\PYG{p}{)}
\PYG{n}{ax2} \PYG{o}{=} \PYG{n}{plt}\PYG{o}{.}\PYG{n}{twinx}\PYG{p}{(}\PYG{n}{ax}\PYG{p}{)}
\PYG{n}{ax2}\PYG{o}{.}\PYG{n}{plot}\PYG{p}{(}\PYG{n}{ts}\PYG{p}{,} \PYG{n}{metabolites}\PYG{p}{[}\PYG{l+s+s2}{\PYGZdq{}}\PYG{l+s+s2}{X\PYGZus{}c}\PYG{l+s+s2}{\PYGZdq{}}\PYG{p}{]}\PYG{p}{,} \PYG{n}{color}\PYG{o}{=}\PYG{l+s+s2}{\PYGZdq{}}\PYG{l+s+s2}{r}\PYG{l+s+s2}{\PYGZdq{}}\PYG{p}{)}

\PYG{n}{ax}\PYG{o}{.}\PYG{n}{set\PYGZus{}ylabel}\PYG{p}{(}\PYG{l+s+s2}{\PYGZdq{}}\PYG{l+s+s2}{Biomass ecoli 1}\PYG{l+s+s2}{\PYGZdq{}}\PYG{p}{,} \PYG{n}{color}\PYG{o}{=}\PYG{l+s+s2}{\PYGZdq{}}\PYG{l+s+s2}{b}\PYG{l+s+s2}{\PYGZdq{}}\PYG{p}{)}
\PYG{n}{ax2}\PYG{o}{.}\PYG{n}{set\PYGZus{}ylabel}\PYG{p}{(}\PYG{l+s+s2}{\PYGZdq{}}\PYG{l+s+s2}{X\PYGZus{}comm}\PYG{l+s+s2}{\PYGZdq{}}\PYG{p}{,} \PYG{n}{color}\PYG{o}{=}\PYG{l+s+s2}{\PYGZdq{}}\PYG{l+s+s2}{r}\PYG{l+s+s2}{\PYGZdq{}}\PYG{p}{)}

\PYG{n}{ax3} \PYG{o}{=} \PYG{n}{plt}\PYG{o}{.}\PYG{n}{twinx}\PYG{p}{(}\PYG{n}{ax}\PYG{p}{)}
\PYG{n}{ax3}\PYG{o}{.}\PYG{n}{plot}\PYG{p}{(}\PYG{n}{ts}\PYG{p}{,} \PYG{n}{biomasses}\PYG{p}{[}\PYG{l+s+s2}{\PYGZdq{}}\PYG{l+s+s2}{ecoli\PYGZus{}2}\PYG{l+s+s2}{\PYGZdq{}}\PYG{p}{]}\PYG{p}{,} \PYG{n}{color}\PYG{o}{=}\PYG{l+s+s2}{\PYGZdq{}}\PYG{l+s+s2}{y}\PYG{l+s+s2}{\PYGZdq{}}\PYG{p}{)}
\PYG{n}{ax3}\PYG{o}{.}\PYG{n}{set\PYGZus{}ylabel}\PYG{p}{(}\PYG{l+s+s2}{\PYGZdq{}}\PYG{l+s+s2}{Biomass ecoli 2}\PYG{l+s+s2}{\PYGZdq{}}\PYG{p}{,} \PYG{n}{color}\PYG{o}{=}\PYG{l+s+s2}{\PYGZdq{}}\PYG{l+s+s2}{y}\PYG{l+s+s2}{\PYGZdq{}}\PYG{p}{)}

\PYG{n}{plt}\PYG{o}{.}\PYG{n}{show}\PYG{p}{(}\PYG{p}{)}
\end{sphinxVerbatim}


\section{Add reaction kinetics}
\label{\detokenize{5_djoint/home:add-reaction-kinetics}}
\sphinxAtStartPar
In construction


\subsection{Write your own kinetics!}
\label{\detokenize{5_djoint/home:write-your-own-kinetics}}
\sphinxAtStartPar
In construction

\sphinxstepscope


\chapter{6. EndpointFBA}
\label{\detokenize{6_endpoint_fba/home:endpointfba}}\label{\detokenize{6_endpoint_fba/home::doc}}
\sphinxstepscope


\chapter{7. FAQ}
\label{\detokenize{7_faq/home:faq}}\label{\detokenize{7_faq/home::doc}}
\sphinxstepscope


\chapter{8. API}
\label{\detokenize{8_API/endPointFBA:api}}\label{\detokenize{8_API/endPointFBA::doc}}

\section{CommunityModel}
\label{\detokenize{8_API/endPointFBA:module-endPointFBA.CommunityModel}}\label{\detokenize{8_API/endPointFBA:communitymodel}}\index{module@\spxentry{module}!endPointFBA.CommunityModel@\spxentry{endPointFBA.CommunityModel}}\index{endPointFBA.CommunityModel@\spxentry{endPointFBA.CommunityModel}!module@\spxentry{module}}\index{CommunityModel (class in endPointFBA.CommunityModel)@\spxentry{CommunityModel}\spxextra{class in endPointFBA.CommunityModel}}

\begin{fulllineitems}
\phantomsection\label{\detokenize{8_API/endPointFBA:endPointFBA.CommunityModel.CommunityModel}}
\pysigstartsignatures
\pysiglinewithargsret{\sphinxbfcode{\sphinxupquote{class\DUrole{w,w}{  }}}\sphinxcode{\sphinxupquote{endPointFBA.CommunityModel.}}\sphinxbfcode{\sphinxupquote{CommunityModel}}}{\sphinxparam{\DUrole{n,n}{models}\DUrole{p,p}{:}\DUrole{w,w}{  }\DUrole{n,n}{list\DUrole{p,p}{{[}}cbmpy.CBModel.Model\DUrole{p,p}{{]}}}}, \sphinxparam{\DUrole{n,n}{biomass\_reaction\_ids}\DUrole{p,p}{:}\DUrole{w,w}{  }\DUrole{n,n}{list\DUrole{p,p}{{[}}str\DUrole{p,p}{{]}}}}, \sphinxparam{\DUrole{n,n}{ids}\DUrole{p,p}{:}\DUrole{w,w}{  }\DUrole{n,n}{list\DUrole{p,p}{{[}}str\DUrole{p,p}{{]}}}\DUrole{w,w}{  }\DUrole{o,o}{=}\DUrole{w,w}{  }\DUrole{default_value}{{[}{]}}}, \sphinxparam{\DUrole{n,n}{combined\_model\_id}\DUrole{p,p}{:}\DUrole{w,w}{  }\DUrole{n,n}{str}\DUrole{w,w}{  }\DUrole{o,o}{=}\DUrole{w,w}{  }\DUrole{default_value}{\textquotesingle{}combined\_model\textquotesingle{}}}}{}
\pysigstopsignatures
\sphinxAtStartPar
Bases: \sphinxcode{\sphinxupquote{Model}}
\index{add\_model\_to\_community() (endPointFBA.CommunityModel.CommunityModel method)@\spxentry{add\_model\_to\_community()}\spxextra{endPointFBA.CommunityModel.CommunityModel method}}

\begin{fulllineitems}
\phantomsection\label{\detokenize{8_API/endPointFBA:endPointFBA.CommunityModel.CommunityModel.add_model_to_community}}
\pysigstartsignatures
\pysiglinewithargsret{\sphinxbfcode{\sphinxupquote{add\_model\_to\_community}}}{\sphinxparam{\DUrole{n,n}{model}\DUrole{p,p}{:}\DUrole{w,w}{  }\DUrole{n,n}{Model}}, \sphinxparam{\DUrole{n,n}{biomass\_reaction}\DUrole{p,p}{:}\DUrole{w,w}{  }\DUrole{n,n}{str}}, \sphinxparam{\DUrole{n,n}{new\_id}\DUrole{p,p}{:}\DUrole{w,w}{  }\DUrole{n,n}{str\DUrole{w,w}{  }\DUrole{p,p}{|}\DUrole{w,w}{  }None}\DUrole{w,w}{  }\DUrole{o,o}{=}\DUrole{w,w}{  }\DUrole{default_value}{None}}}{}
\pysigstopsignatures
\sphinxAtStartPar
Adds a model to the CommunityModel
\begin{description}
\sphinxlineitem{Args:}
\sphinxAtStartPar
model (Model): The model that needs to be added
biomass\_reaction (str): The reaction id of the biomass reaction of
the new model
new\_id (str, optional): The user set identifier. Defaults to None.
If set to None the model.id will be used

\end{description}

\end{fulllineitems}

\index{get\_model\_biomass\_ids() (endPointFBA.CommunityModel.CommunityModel method)@\spxentry{get\_model\_biomass\_ids()}\spxextra{endPointFBA.CommunityModel.CommunityModel method}}

\begin{fulllineitems}
\phantomsection\label{\detokenize{8_API/endPointFBA:endPointFBA.CommunityModel.CommunityModel.get_model_biomass_ids}}
\pysigstartsignatures
\pysiglinewithargsret{\sphinxbfcode{\sphinxupquote{get\_model\_biomass\_ids}}}{}{{ $\rightarrow$ dict\DUrole{p,p}{{[}}str\DUrole{p,p}{,}\DUrole{w,w}{  }str\DUrole{p,p}{{]}}}}
\pysigstopsignatures
\end{fulllineitems}

\index{get\_model\_specific\_reactions() (endPointFBA.CommunityModel.CommunityModel method)@\spxentry{get\_model\_specific\_reactions()}\spxextra{endPointFBA.CommunityModel.CommunityModel method}}

\begin{fulllineitems}
\phantomsection\label{\detokenize{8_API/endPointFBA:endPointFBA.CommunityModel.CommunityModel.get_model_specific_reactions}}
\pysigstartsignatures
\pysiglinewithargsret{\sphinxbfcode{\sphinxupquote{get\_model\_specific\_reactions}}}{\sphinxparam{\DUrole{n,n}{mid}\DUrole{p,p}{:}\DUrole{w,w}{  }\DUrole{n,n}{str}}}{{ $\rightarrow$ list\DUrole{p,p}{{[}}cbmpy.CBModel.Reaction\DUrole{p,p}{{]}}}}
\pysigstopsignatures
\sphinxAtStartPar
Returns a list of reaction ids
\begin{description}
\sphinxlineitem{Args:}
\sphinxAtStartPar
mid (str): \_description\_

\sphinxlineitem{Raises:}
\sphinxAtStartPar
NotInCombinedModel: \_description\_

\sphinxlineitem{Returns:}
\sphinxAtStartPar
list{[}Reaction{]}: \_description\_

\end{description}

\end{fulllineitems}

\index{get\_model\_specific\_species() (endPointFBA.CommunityModel.CommunityModel method)@\spxentry{get\_model\_specific\_species()}\spxextra{endPointFBA.CommunityModel.CommunityModel method}}

\begin{fulllineitems}
\phantomsection\label{\detokenize{8_API/endPointFBA:endPointFBA.CommunityModel.CommunityModel.get_model_specific_species}}
\pysigstartsignatures
\pysiglinewithargsret{\sphinxbfcode{\sphinxupquote{get\_model\_specific\_species}}}{\sphinxparam{\DUrole{n,n}{mid}\DUrole{p,p}{:}\DUrole{w,w}{  }\DUrole{n,n}{str}}}{{ $\rightarrow$ list\DUrole{p,p}{{[}}cbmpy.CBModel.Species\DUrole{p,p}{{]}}}}
\pysigstopsignatures
\end{fulllineitems}

\index{get\_reaction\_bigg\_ids() (endPointFBA.CommunityModel.CommunityModel method)@\spxentry{get\_reaction\_bigg\_ids()}\spxextra{endPointFBA.CommunityModel.CommunityModel method}}

\begin{fulllineitems}
\phantomsection\label{\detokenize{8_API/endPointFBA:endPointFBA.CommunityModel.CommunityModel.get_reaction_bigg_ids}}
\pysigstartsignatures
\pysiglinewithargsret{\sphinxbfcode{\sphinxupquote{get\_reaction\_bigg\_ids}}}{\sphinxparam{\DUrole{n,n}{mid}\DUrole{o,o}{=}\DUrole{default_value}{\textquotesingle{}\textquotesingle{}}}}{{ $\rightarrow$ list\DUrole{p,p}{{[}}str\DUrole{p,p}{{]}}}}
\pysigstopsignatures
\sphinxAtStartPar
Get the reaction bigg ids of all reactions
\begin{description}
\sphinxlineitem{Args:}
\sphinxAtStartPar
mid (str, optional): If a model id is provided only reactions from
the specific model are returned
Defaults to “”.

\sphinxlineitem{Raises:}
\sphinxAtStartPar
NotInCombinedModel: the id provided was not in the combined model

\sphinxlineitem{Returns:}
\sphinxAtStartPar
list{[}str{]}: list containing the bigg ids

\end{description}

\end{fulllineitems}

\index{get\_species\_bigg\_ids() (endPointFBA.CommunityModel.CommunityModel method)@\spxentry{get\_species\_bigg\_ids()}\spxextra{endPointFBA.CommunityModel.CommunityModel method}}

\begin{fulllineitems}
\phantomsection\label{\detokenize{8_API/endPointFBA:endPointFBA.CommunityModel.CommunityModel.get_species_bigg_ids}}
\pysigstartsignatures
\pysiglinewithargsret{\sphinxbfcode{\sphinxupquote{get\_species\_bigg\_ids}}}{\sphinxparam{\DUrole{n,n}{mid}\DUrole{o,o}{=}\DUrole{default_value}{\textquotesingle{}\textquotesingle{}}}}{{ $\rightarrow$ list\DUrole{p,p}{{[}}str\DUrole{p,p}{{]}}}}
\pysigstopsignatures
\sphinxAtStartPar
Get the species bigg ids of all species
\begin{description}
\sphinxlineitem{Args:}
\sphinxAtStartPar
mid (str, optional): When provided only the ids of a specific
model are returned.
Defaults to “”.

\sphinxlineitem{Raises:}
\sphinxAtStartPar
NotInCombinedModel: the id provided was not in the combined model

\sphinxlineitem{Returns:}
\sphinxAtStartPar
list{[}str{]}: list containing the bigg ids

\end{description}

\end{fulllineitems}

\index{identify\_biomass\_of\_model\_from\_reaction\_id() (endPointFBA.CommunityModel.CommunityModel method)@\spxentry{identify\_biomass\_of\_model\_from\_reaction\_id()}\spxextra{endPointFBA.CommunityModel.CommunityModel method}}

\begin{fulllineitems}
\phantomsection\label{\detokenize{8_API/endPointFBA:endPointFBA.CommunityModel.CommunityModel.identify_biomass_of_model_from_reaction_id}}
\pysigstartsignatures
\pysiglinewithargsret{\sphinxbfcode{\sphinxupquote{identify\_biomass\_of\_model\_from\_reaction\_id}}}{\sphinxparam{\DUrole{n,n}{rid}}}{{ $\rightarrow$ str}}
\pysigstopsignatures
\end{fulllineitems}

\index{identify\_biomass\_reaction\_for\_model() (endPointFBA.CommunityModel.CommunityModel method)@\spxentry{identify\_biomass\_reaction\_for\_model()}\spxextra{endPointFBA.CommunityModel.CommunityModel method}}

\begin{fulllineitems}
\phantomsection\label{\detokenize{8_API/endPointFBA:endPointFBA.CommunityModel.CommunityModel.identify_biomass_reaction_for_model}}
\pysigstartsignatures
\pysiglinewithargsret{\sphinxbfcode{\sphinxupquote{identify\_biomass\_reaction\_for\_model}}}{\sphinxparam{\DUrole{n,n}{mid}\DUrole{p,p}{:}\DUrole{w,w}{  }\DUrole{n,n}{str}}}{{ $\rightarrow$ list\DUrole{p,p}{{[}}str\DUrole{p,p}{{]}}}}
\pysigstopsignatures
\sphinxAtStartPar
Given a model id return the biomass reaction
\begin{description}
\sphinxlineitem{Args:}
\sphinxAtStartPar
mid (str): \_description\_

\sphinxlineitem{Returns:}
\sphinxAtStartPar
str: \_description\_

\end{description}

\end{fulllineitems}

\index{identify\_model\_from\_reaction() (endPointFBA.CommunityModel.CommunityModel method)@\spxentry{identify\_model\_from\_reaction()}\spxextra{endPointFBA.CommunityModel.CommunityModel method}}

\begin{fulllineitems}
\phantomsection\label{\detokenize{8_API/endPointFBA:endPointFBA.CommunityModel.CommunityModel.identify_model_from_reaction}}
\pysigstartsignatures
\pysiglinewithargsret{\sphinxbfcode{\sphinxupquote{identify\_model\_from\_reaction}}}{\sphinxparam{\DUrole{n,n}{rid}\DUrole{p,p}{:}\DUrole{w,w}{  }\DUrole{n,n}{str}}}{{ $\rightarrow$ str}}
\pysigstopsignatures
\sphinxAtStartPar
Given a reaction id get the single model this reaction belonged to
\begin{description}
\sphinxlineitem{Args:}
\sphinxAtStartPar
rid (str): reaction id of the kinetic model

\sphinxlineitem{Returns:}
\sphinxAtStartPar
str: id of the old model

\end{description}

\end{fulllineitems}

\index{m\_identifiers (endPointFBA.CommunityModel.CommunityModel attribute)@\spxentry{m\_identifiers}\spxextra{endPointFBA.CommunityModel.CommunityModel attribute}}

\begin{fulllineitems}
\phantomsection\label{\detokenize{8_API/endPointFBA:endPointFBA.CommunityModel.CommunityModel.m_identifiers}}
\pysigstartsignatures
\pysigline{\sphinxbfcode{\sphinxupquote{m\_identifiers}}\sphinxbfcode{\sphinxupquote{\DUrole{p,p}{:}\DUrole{w,w}{  }list\DUrole{p,p}{{[}}str\DUrole{p,p}{{]}}}}}
\pysigstopsignatures
\end{fulllineitems}

\index{m\_single\_model\_biomass\_reaction\_ids (endPointFBA.CommunityModel.CommunityModel attribute)@\spxentry{m\_single\_model\_biomass\_reaction\_ids}\spxextra{endPointFBA.CommunityModel.CommunityModel attribute}}

\begin{fulllineitems}
\phantomsection\label{\detokenize{8_API/endPointFBA:endPointFBA.CommunityModel.CommunityModel.m_single_model_biomass_reaction_ids}}
\pysigstartsignatures
\pysigline{\sphinxbfcode{\sphinxupquote{m\_single\_model\_biomass\_reaction\_ids}}\sphinxbfcode{\sphinxupquote{\DUrole{p,p}{:}\DUrole{w,w}{  }list\DUrole{p,p}{{[}}str\DUrole{p,p}{{]}}}}\sphinxbfcode{\sphinxupquote{\DUrole{w,w}{  }\DUrole{p,p}{=}\DUrole{w,w}{  }{[}{]}}}}
\pysigstopsignatures
\end{fulllineitems}

\index{m\_single\_model\_identifiers (endPointFBA.CommunityModel.CommunityModel attribute)@\spxentry{m\_single\_model\_identifiers}\spxextra{endPointFBA.CommunityModel.CommunityModel attribute}}

\begin{fulllineitems}
\phantomsection\label{\detokenize{8_API/endPointFBA:endPointFBA.CommunityModel.CommunityModel.m_single_model_identifiers}}
\pysigstartsignatures
\pysigline{\sphinxbfcode{\sphinxupquote{m\_single\_model\_identifiers}}\sphinxbfcode{\sphinxupquote{\DUrole{p,p}{:}\DUrole{w,w}{  }list\DUrole{p,p}{{[}}str\DUrole{p,p}{{]}}}}}
\pysigstopsignatures
\end{fulllineitems}


\end{fulllineitems}



\section{KineticModel}
\label{\detokenize{8_API/endPointFBA:module-endPointFBA.KineticModel}}\label{\detokenize{8_API/endPointFBA:kineticmodel}}\index{module@\spxentry{module}!endPointFBA.KineticModel@\spxentry{endPointFBA.KineticModel}}\index{endPointFBA.KineticModel@\spxentry{endPointFBA.KineticModel}!module@\spxentry{module}}\index{KineticStruct (class in endPointFBA.KineticModel)@\spxentry{KineticStruct}\spxextra{class in endPointFBA.KineticModel}}

\begin{fulllineitems}
\phantomsection\label{\detokenize{8_API/endPointFBA:endPointFBA.KineticModel.KineticStruct}}
\pysigstartsignatures
\pysiglinewithargsret{\sphinxbfcode{\sphinxupquote{class\DUrole{w,w}{  }}}\sphinxcode{\sphinxupquote{endPointFBA.KineticModel.}}\sphinxbfcode{\sphinxupquote{KineticStruct}}}{\sphinxparam{\DUrole{n,n}{kinetics}\DUrole{p,p}{:}\DUrole{w,w}{  }\DUrole{n,n}{dict\DUrole{p,p}{{[}}str\DUrole{p,p}{,}\DUrole{w,w}{  }tuple\DUrole{p,p}{{[}}float\DUrole{p,p}{,}\DUrole{w,w}{  }float\DUrole{p,p}{{]}}\DUrole{p,p}{{]}}}}}{}
\pysigstopsignatures
\sphinxAtStartPar
Bases: \sphinxcode{\sphinxupquote{object}}
\index{get\_model() (endPointFBA.KineticModel.KineticStruct method)@\spxentry{get\_model()}\spxextra{endPointFBA.KineticModel.KineticStruct method}}

\begin{fulllineitems}
\phantomsection\label{\detokenize{8_API/endPointFBA:endPointFBA.KineticModel.KineticStruct.get_model}}
\pysigstartsignatures
\pysiglinewithargsret{\sphinxbfcode{\sphinxupquote{get\_model}}}{}{{ $\rightarrow$ Model}}
\pysigstopsignatures
\end{fulllineitems}

\index{get\_model\_kinetics() (endPointFBA.KineticModel.KineticStruct method)@\spxentry{get\_model\_kinetics()}\spxextra{endPointFBA.KineticModel.KineticStruct method}}

\begin{fulllineitems}
\phantomsection\label{\detokenize{8_API/endPointFBA:endPointFBA.KineticModel.KineticStruct.get_model_kinetics}}
\pysigstartsignatures
\pysiglinewithargsret{\sphinxbfcode{\sphinxupquote{get\_model\_kinetics}}}{}{{ $\rightarrow$ dict\DUrole{p,p}{{[}}str\DUrole{p,p}{,}\DUrole{w,w}{  }tuple\DUrole{p,p}{{[}}float\DUrole{p,p}{,}\DUrole{w,w}{  }float\DUrole{p,p}{{]}}\DUrole{p,p}{{]}}}}
\pysigstopsignatures
\end{fulllineitems}

\index{get\_reaction\_kinetics() (endPointFBA.KineticModel.KineticStruct method)@\spxentry{get\_reaction\_kinetics()}\spxextra{endPointFBA.KineticModel.KineticStruct method}}

\begin{fulllineitems}
\phantomsection\label{\detokenize{8_API/endPointFBA:endPointFBA.KineticModel.KineticStruct.get_reaction_kinetics}}
\pysigstartsignatures
\pysiglinewithargsret{\sphinxbfcode{\sphinxupquote{get\_reaction\_kinetics}}}{\sphinxparam{\DUrole{n,n}{rid}}}{{ $\rightarrow$ tuple\DUrole{p,p}{{[}}float\DUrole{p,p}{,}\DUrole{w,w}{  }float\DUrole{p,p}{{]}}}}
\pysigstopsignatures
\end{fulllineitems}

\index{get\_reaction\_km() (endPointFBA.KineticModel.KineticStruct method)@\spxentry{get\_reaction\_km()}\spxextra{endPointFBA.KineticModel.KineticStruct method}}

\begin{fulllineitems}
\phantomsection\label{\detokenize{8_API/endPointFBA:endPointFBA.KineticModel.KineticStruct.get_reaction_km}}
\pysigstartsignatures
\pysiglinewithargsret{\sphinxbfcode{\sphinxupquote{get\_reaction\_km}}}{\sphinxparam{\DUrole{n,n}{rid}}}{{ $\rightarrow$ float}}
\pysigstopsignatures
\end{fulllineitems}

\index{get\_reaction\_vmax() (endPointFBA.KineticModel.KineticStruct method)@\spxentry{get\_reaction\_vmax()}\spxextra{endPointFBA.KineticModel.KineticStruct method}}

\begin{fulllineitems}
\phantomsection\label{\detokenize{8_API/endPointFBA:endPointFBA.KineticModel.KineticStruct.get_reaction_vmax}}
\pysigstartsignatures
\pysiglinewithargsret{\sphinxbfcode{\sphinxupquote{get\_reaction\_vmax}}}{\sphinxparam{\DUrole{n,n}{rid}}}{{ $\rightarrow$ float}}
\pysigstopsignatures
\end{fulllineitems}

\index{m\_kinetics (endPointFBA.KineticModel.KineticStruct attribute)@\spxentry{m\_kinetics}\spxextra{endPointFBA.KineticModel.KineticStruct attribute}}

\begin{fulllineitems}
\phantomsection\label{\detokenize{8_API/endPointFBA:endPointFBA.KineticModel.KineticStruct.m_kinetics}}
\pysigstartsignatures
\pysigline{\sphinxbfcode{\sphinxupquote{m\_kinetics}}\sphinxbfcode{\sphinxupquote{\DUrole{p,p}{:}\DUrole{w,w}{  }dict\DUrole{p,p}{{[}}str\DUrole{p,p}{,}\DUrole{w,w}{  }tuple\DUrole{p,p}{{[}}float\DUrole{p,p}{,}\DUrole{w,w}{  }float\DUrole{p,p}{{]}}\DUrole{p,p}{{]}}}}}
\pysigstopsignatures
\end{fulllineitems}

\index{set\_kinetics() (endPointFBA.KineticModel.KineticStruct method)@\spxentry{set\_kinetics()}\spxextra{endPointFBA.KineticModel.KineticStruct method}}

\begin{fulllineitems}
\phantomsection\label{\detokenize{8_API/endPointFBA:endPointFBA.KineticModel.KineticStruct.set_kinetics}}
\pysigstartsignatures
\pysiglinewithargsret{\sphinxbfcode{\sphinxupquote{set\_kinetics}}}{\sphinxparam{\DUrole{n,n}{kinetics}\DUrole{p,p}{:}\DUrole{w,w}{  }\DUrole{n,n}{dict\DUrole{p,p}{{[}}str\DUrole{p,p}{,}\DUrole{w,w}{  }tuple\DUrole{p,p}{{[}}float\DUrole{p,p}{,}\DUrole{w,w}{  }float\DUrole{p,p}{{]}}\DUrole{p,p}{{]}}}}}{}
\pysigstopsignatures
\end{fulllineitems}

\index{set\_model() (endPointFBA.KineticModel.KineticStruct method)@\spxentry{set\_model()}\spxextra{endPointFBA.KineticModel.KineticStruct method}}

\begin{fulllineitems}
\phantomsection\label{\detokenize{8_API/endPointFBA:endPointFBA.KineticModel.KineticStruct.set_model}}
\pysigstartsignatures
\pysiglinewithargsret{\sphinxbfcode{\sphinxupquote{set\_model}}}{\sphinxparam{\DUrole{n,n}{model}\DUrole{p,p}{:}\DUrole{w,w}{  }\DUrole{n,n}{Model}}}{{ $\rightarrow$ None}}
\pysigstopsignatures
\end{fulllineitems}

\index{set\_reaction\_kinetics() (endPointFBA.KineticModel.KineticStruct method)@\spxentry{set\_reaction\_kinetics()}\spxextra{endPointFBA.KineticModel.KineticStruct method}}

\begin{fulllineitems}
\phantomsection\label{\detokenize{8_API/endPointFBA:endPointFBA.KineticModel.KineticStruct.set_reaction_kinetics}}
\pysigstartsignatures
\pysiglinewithargsret{\sphinxbfcode{\sphinxupquote{set\_reaction\_kinetics}}}{\sphinxparam{\DUrole{n,n}{rid}\DUrole{p,p}{:}\DUrole{w,w}{  }\DUrole{n,n}{str}}, \sphinxparam{\DUrole{n,n}{kinetics}\DUrole{p,p}{:}\DUrole{w,w}{  }\DUrole{n,n}{tuple\DUrole{p,p}{{[}}float\DUrole{p,p}{,}\DUrole{w,w}{  }float\DUrole{p,p}{{]}}}}}{}
\pysigstopsignatures
\end{fulllineitems}


\end{fulllineitems}



\section{DynamicJointFba}
\label{\detokenize{8_API/endPointFBA:module-endPointFBA.DynamicJointFBA}}\label{\detokenize{8_API/endPointFBA:dynamicjointfba}}\index{module@\spxentry{module}!endPointFBA.DynamicJointFBA@\spxentry{endPointFBA.DynamicJointFBA}}\index{endPointFBA.DynamicJointFBA@\spxentry{endPointFBA.DynamicJointFBA}!module@\spxentry{module}}\index{DynamicJointFBA (class in endPointFBA.DynamicJointFBA)@\spxentry{DynamicJointFBA}\spxextra{class in endPointFBA.DynamicJointFBA}}

\begin{fulllineitems}
\phantomsection\label{\detokenize{8_API/endPointFBA:endPointFBA.DynamicJointFBA.DynamicJointFBA}}
\pysigstartsignatures
\pysiglinewithargsret{\sphinxbfcode{\sphinxupquote{class\DUrole{w,w}{  }}}\sphinxcode{\sphinxupquote{endPointFBA.DynamicJointFBA.}}\sphinxbfcode{\sphinxupquote{DynamicJointFBA}}}{\sphinxparam{\DUrole{n,n}{model}\DUrole{p,p}{:}\DUrole{w,w}{  }\DUrole{n,n}{{\hyperref[\detokenize{8_API/endPointFBA:endPointFBA.CommunityModel.CommunityModel}]{\sphinxcrossref{CommunityModel}}}}}, \sphinxparam{\DUrole{n,n}{biomasses}\DUrole{p,p}{:}\DUrole{w,w}{  }\DUrole{n,n}{list\DUrole{p,p}{{[}}float\DUrole{p,p}{{]}}}}, \sphinxparam{\DUrole{n,n}{initial\_concentrations}\DUrole{p,p}{:}\DUrole{w,w}{  }\DUrole{n,n}{dict\DUrole{p,p}{{[}}str\DUrole{p,p}{,}\DUrole{w,w}{  }float\DUrole{p,p}{{]}}}\DUrole{w,w}{  }\DUrole{o,o}{=}\DUrole{w,w}{  }\DUrole{default_value}{\{\}}}, \sphinxparam{\DUrole{n,n}{kinetics}\DUrole{o,o}{=}\DUrole{default_value}{\{\}}}}{}
\pysigstopsignatures
\sphinxAtStartPar
Bases: \sphinxcode{\sphinxupquote{DynamicFBABase}}
\index{get\_joint\_model() (endPointFBA.DynamicJointFBA.DynamicJointFBA method)@\spxentry{get\_joint\_model()}\spxextra{endPointFBA.DynamicJointFBA.DynamicJointFBA method}}

\begin{fulllineitems}
\phantomsection\label{\detokenize{8_API/endPointFBA:endPointFBA.DynamicJointFBA.DynamicJointFBA.get_joint_model}}
\pysigstartsignatures
\pysiglinewithargsret{\sphinxbfcode{\sphinxupquote{get\_joint\_model}}}{}{{ $\rightarrow$ {\hyperref[\detokenize{8_API/endPointFBA:endPointFBA.CommunityModel.CommunityModel}]{\sphinxcrossref{CommunityModel}}}}}
\pysigstopsignatures
\end{fulllineitems}

\index{m\_exporters (endPointFBA.DynamicJointFBA.DynamicJointFBA attribute)@\spxentry{m\_exporters}\spxextra{endPointFBA.DynamicJointFBA.DynamicJointFBA attribute}}

\begin{fulllineitems}
\phantomsection\label{\detokenize{8_API/endPointFBA:endPointFBA.DynamicJointFBA.DynamicJointFBA.m_exporters}}
\pysigstartsignatures
\pysigline{\sphinxbfcode{\sphinxupquote{m\_exporters}}\sphinxbfcode{\sphinxupquote{\DUrole{p,p}{:}\DUrole{w,w}{  }dict\DUrole{p,p}{{[}}str\DUrole{p,p}{,}\DUrole{w,w}{  }list\DUrole{p,p}{{[}}str\DUrole{p,p}{{]}}\DUrole{p,p}{{]}}}}}
\pysigstopsignatures
\end{fulllineitems}

\index{m\_importers (endPointFBA.DynamicJointFBA.DynamicJointFBA attribute)@\spxentry{m\_importers}\spxextra{endPointFBA.DynamicJointFBA.DynamicJointFBA attribute}}

\begin{fulllineitems}
\phantomsection\label{\detokenize{8_API/endPointFBA:endPointFBA.DynamicJointFBA.DynamicJointFBA.m_importers}}
\pysigstartsignatures
\pysigline{\sphinxbfcode{\sphinxupquote{m\_importers}}\sphinxbfcode{\sphinxupquote{\DUrole{p,p}{:}\DUrole{w,w}{  }dict\DUrole{p,p}{{[}}str\DUrole{p,p}{,}\DUrole{w,w}{  }list\DUrole{p,p}{{[}}str\DUrole{p,p}{{]}}\DUrole{p,p}{{]}}}}}
\pysigstopsignatures
\end{fulllineitems}

\index{m\_initial\_bounds (endPointFBA.DynamicJointFBA.DynamicJointFBA attribute)@\spxentry{m\_initial\_bounds}\spxextra{endPointFBA.DynamicJointFBA.DynamicJointFBA attribute}}

\begin{fulllineitems}
\phantomsection\label{\detokenize{8_API/endPointFBA:endPointFBA.DynamicJointFBA.DynamicJointFBA.m_initial_bounds}}
\pysigstartsignatures
\pysigline{\sphinxbfcode{\sphinxupquote{m\_initial\_bounds}}\sphinxbfcode{\sphinxupquote{\DUrole{p,p}{:}\DUrole{w,w}{  }dict\DUrole{p,p}{{[}}str\DUrole{p,p}{,}\DUrole{w,w}{  }tuple\DUrole{p,p}{{[}}float\DUrole{p,p}{,}\DUrole{w,w}{  }float\DUrole{p,p}{{]}}\DUrole{p,p}{{]}}}}\sphinxbfcode{\sphinxupquote{\DUrole{w,w}{  }\DUrole{p,p}{=}\DUrole{w,w}{  }\{\}}}}
\pysigstopsignatures
\end{fulllineitems}

\index{m\_kinetics (endPointFBA.DynamicJointFBA.DynamicJointFBA attribute)@\spxentry{m\_kinetics}\spxextra{endPointFBA.DynamicJointFBA.DynamicJointFBA attribute}}

\begin{fulllineitems}
\phantomsection\label{\detokenize{8_API/endPointFBA:endPointFBA.DynamicJointFBA.DynamicJointFBA.m_kinetics}}
\pysigstartsignatures
\pysigline{\sphinxbfcode{\sphinxupquote{m\_kinetics}}\sphinxbfcode{\sphinxupquote{\DUrole{p,p}{:}\DUrole{w,w}{  }dict\DUrole{p,p}{{[}}str\DUrole{p,p}{,}\DUrole{w,w}{  }tuple\DUrole{p,p}{{[}}float\DUrole{p,p}{,}\DUrole{w,w}{  }float\DUrole{p,p}{{]}}\DUrole{p,p}{{]}}}}}
\pysigstopsignatures
\end{fulllineitems}

\index{m\_model (endPointFBA.DynamicJointFBA.DynamicJointFBA attribute)@\spxentry{m\_model}\spxextra{endPointFBA.DynamicJointFBA.DynamicJointFBA attribute}}

\begin{fulllineitems}
\phantomsection\label{\detokenize{8_API/endPointFBA:endPointFBA.DynamicJointFBA.DynamicJointFBA.m_model}}
\pysigstartsignatures
\pysigline{\sphinxbfcode{\sphinxupquote{m\_model}}\sphinxbfcode{\sphinxupquote{\DUrole{p,p}{:}\DUrole{w,w}{  }{\hyperref[\detokenize{8_API/endPointFBA:endPointFBA.CommunityModel.CommunityModel}]{\sphinxcrossref{CommunityModel}}}}}}
\pysigstopsignatures
\end{fulllineitems}

\index{set\_community\_biomass\_reaction() (endPointFBA.DynamicJointFBA.DynamicJointFBA method)@\spxentry{set\_community\_biomass\_reaction()}\spxextra{endPointFBA.DynamicJointFBA.DynamicJointFBA method}}

\begin{fulllineitems}
\phantomsection\label{\detokenize{8_API/endPointFBA:endPointFBA.DynamicJointFBA.DynamicJointFBA.set_community_biomass_reaction}}
\pysigstartsignatures
\pysiglinewithargsret{\sphinxbfcode{\sphinxupquote{set\_community\_biomass\_reaction}}}{\sphinxparam{\DUrole{n,n}{model}\DUrole{p,p}{:}\DUrole{w,w}{  }\DUrole{n,n}{{\hyperref[\detokenize{8_API/endPointFBA:endPointFBA.CommunityModel.CommunityModel}]{\sphinxcrossref{CommunityModel}}}}}}{}
\pysigstopsignatures
\end{fulllineitems}

\index{simulate() (endPointFBA.DynamicJointFBA.DynamicJointFBA method)@\spxentry{simulate()}\spxextra{endPointFBA.DynamicJointFBA.DynamicJointFBA method}}

\begin{fulllineitems}
\phantomsection\label{\detokenize{8_API/endPointFBA:endPointFBA.DynamicJointFBA.DynamicJointFBA.simulate}}
\pysigstartsignatures
\pysiglinewithargsret{\sphinxbfcode{\sphinxupquote{simulate}}}{\sphinxparam{\DUrole{n,n}{dt}\DUrole{p,p}{:}\DUrole{w,w}{  }\DUrole{n,n}{float}}, \sphinxparam{\DUrole{n,n}{epsilon}\DUrole{o,o}{=}\DUrole{default_value}{0.001}}, \sphinxparam{\DUrole{n,n}{user\_func}\DUrole{o,o}{=}\DUrole{default_value}{None}}}{}
\pysigstopsignatures
\end{fulllineitems}

\index{update\_bounds() (endPointFBA.DynamicJointFBA.DynamicJointFBA method)@\spxentry{update\_bounds()}\spxextra{endPointFBA.DynamicJointFBA.DynamicJointFBA method}}

\begin{fulllineitems}
\phantomsection\label{\detokenize{8_API/endPointFBA:endPointFBA.DynamicJointFBA.DynamicJointFBA.update_bounds}}
\pysigstartsignatures
\pysiglinewithargsret{\sphinxbfcode{\sphinxupquote{update\_bounds}}}{\sphinxparam{\DUrole{n,n}{user\_func}}}{{ $\rightarrow$ None}}
\pysigstopsignatures
\end{fulllineitems}

\index{update\_concentrations() (endPointFBA.DynamicJointFBA.DynamicJointFBA method)@\spxentry{update\_concentrations()}\spxextra{endPointFBA.DynamicJointFBA.DynamicJointFBA method}}

\begin{fulllineitems}
\phantomsection\label{\detokenize{8_API/endPointFBA:endPointFBA.DynamicJointFBA.DynamicJointFBA.update_concentrations}}
\pysigstartsignatures
\pysiglinewithargsret{\sphinxbfcode{\sphinxupquote{update\_concentrations}}}{\sphinxparam{\DUrole{n,n}{FBAsol}}, \sphinxparam{\DUrole{n,n}{dt}}}{}
\pysigstopsignatures
\end{fulllineitems}

\index{update\_importer\_bounds() (endPointFBA.DynamicJointFBA.DynamicJointFBA method)@\spxentry{update\_importer\_bounds()}\spxextra{endPointFBA.DynamicJointFBA.DynamicJointFBA method}}

\begin{fulllineitems}
\phantomsection\label{\detokenize{8_API/endPointFBA:endPointFBA.DynamicJointFBA.DynamicJointFBA.update_importer_bounds}}
\pysigstartsignatures
\pysiglinewithargsret{\sphinxbfcode{\sphinxupquote{update\_importer\_bounds}}}{\sphinxparam{\DUrole{n,n}{reaction}\DUrole{p,p}{:}\DUrole{w,w}{  }\DUrole{n,n}{Reaction}}, \sphinxparam{\DUrole{n,n}{X}\DUrole{p,p}{:}\DUrole{w,w}{  }\DUrole{n,n}{float}}}{}
\pysigstopsignatures
\end{fulllineitems}


\end{fulllineitems}



\section{Module contents}
\label{\detokenize{8_API/endPointFBA:module-endPointFBA}}\label{\detokenize{8_API/endPointFBA:module-contents}}\index{module@\spxentry{module}!endPointFBA@\spxentry{endPointFBA}}\index{endPointFBA@\spxentry{endPointFBA}!module@\spxentry{module}}

\renewcommand{\indexname}{Python Module Index}
\begin{sphinxtheindex}
\let\bigletter\sphinxstyleindexlettergroup
\bigletter{e}
\item\relax\sphinxstyleindexentry{endPointFBA}\sphinxstyleindexpageref{8_API/endPointFBA:\detokenize{module-endPointFBA}}
\item\relax\sphinxstyleindexentry{endPointFBA.CommunityModel}\sphinxstyleindexpageref{8_API/endPointFBA:\detokenize{module-endPointFBA.CommunityModel}}
\item\relax\sphinxstyleindexentry{endPointFBA.DynamicJointFBA}\sphinxstyleindexpageref{8_API/endPointFBA:\detokenize{module-endPointFBA.DynamicJointFBA}}
\item\relax\sphinxstyleindexentry{endPointFBA.KineticModel}\sphinxstyleindexpageref{8_API/endPointFBA:\detokenize{module-endPointFBA.KineticModel}}
\end{sphinxtheindex}

\renewcommand{\indexname}{Index}
\printindex
\end{document}